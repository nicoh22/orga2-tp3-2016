\subsection{Interrupciones}

Para poder atender interrupciones primero debemos llenar la IDT y lo conseguimos llamando a la funcion \verb|idt_inicializar|. 

En ella utilizamos un macro en el cual referimos cada entrada $N$ al segmento de código de kernel, con el offset correspondiente su rutina de atención \_isr$N$. 

Todas las rutinas de atención de interrupciones deben correr con nivel de privilegio cero, por lo que en sus descriptores usamos el selector de segmento correspondiente al indice 4 de la GDT, el segmento de código de kernel.
Además, no deben poder ser invocadas por código de usuario, por lo que las definimos con DPL 0. La única excepción es la interrupción \textit{0x66}, que utilizamos para las syscalls a disposición de los usuarios. Por este motivo, su entrada en la IDT especifica DPL 3.
Independientemente de su nivel de privilegios, definimos todas las interrupciones con tipo \textit{interrupt}.\\



\subsubsection{Excepciones}

En un principio, resolvimos el manejo de excepciones definiendo los primeros veinte indices (de 0 a 19) en la tabla de descriptores de interrupción. 
Para atender excepciones escribimos rutinas que muestran sus mensajes de error por pantalla.
En pos de esto, guardamos los mensajes de error correspondientes en cada posición del arreglo \verb|mensajesExcepcion|.
Luego, definimos cada \_isr$N$ con un macro en el cual obtenemos y mostramos el $N$-ésimo mensaje de \verb|mensajesExcepcion|.\\

Este comportamiento luego fue revisado para la implementación del desalojo de tareas y el modo debug. En su versión final, cada rutina de excepción muestra la pantalla de debug si esta activado, imprime el mensaje de error y desaloja la tarea causante (ver secciones \ref{sec-desalojo} y \ref{sec:debug}).\\

\newpage


\subsubsection{Interrupción de reloj}

\label{subsec:int-reloj}

Cuando nos llega un tick de reloj desde el PIC, actualizamos el reloj de sistema situado en la esquina inferior derecha de la pantalla mediante la funcion \verb|proximo_reloj|. Luego pedimos al scheduler el indice de la gdt de la proxima tarea a ejecutar. Logramos esto llamando a la funcion \verb|sched_proximo_indice| (ver seccion \ref{sec:sched}); en caso de devolvernos 0, interpretamos que no debemos cambiar de tarea, sino saltamos usando el indice de gdt obtenido.

Tanto si cambiamos de tarea como si no, se llama a la función \verb|game_tick|, una función que por conveniencia decidimos escribirla en C, que se encarga de mantener la consistencia entre el estado del juego y los datos que se muestran en pantalla. 
Esto implica volver a pintar el fondo si se sale de modo debug (ver sección \ref{sec:debug}), pintar las vidas y puntos actuales de cada jugador y ademas pintar las tareas, p\'aginas mapeadas y actualizar sus relojes.



\subsubsection{Interrupciones de teclado}
\label{subsec:int-teclado}
El handler de interrupción de teclado solo lee el puerto del teclado ($0x60$) y se lo pasa a la función \verb|atender_teclado|. Esta decide el curso de accion en base a la tecla presionada. 

Las teclas 'w', 'a' , 's' y 'd' mueven al jugador 1 y las teclas 'i', 'j', 'k' y 'l' al jugador 2. Las teclas 'shift' permiten a los jugadores lanzar tareas. Las funciones que implementan estos comportamientos se detallan en la sección \ref{sec:game}. 

Sin embargo estas teclas son ignoradas si se produjo una excepción mientras el sistema se encontraba en modo debug. La tecla 'y' nos permite en todo momento activar o desactivar el modo debug.


\subsubsection{Servicios de sistema}

El handler de esta interrupción pasa como parámetros a la función \verb|manejar_syscall| los registros eax, ebx y ecx. Esta funcion setea el parametro global \verb|en_idle| ya que tanto en el caso de ejecutarse exitosamente la operacion requerida como no, debemos poner a correr la tarea idle por el resto del tiempo asignado a la tarea que llamó a la syscall. Luego chequeamos si la tarea nos paso correctamente el parametro \verb|syscall|, de no ser asi la tarea es desalojada. Caso contrario llamamos a la funcion correspondiente al servicio pedido. Estas funciones son responsables de verificar el/los parametros recibidos.
Para mas información sobre  las funciones encontradas en \verb|manejar_syscall| ver la sección \ref{sec:game}

