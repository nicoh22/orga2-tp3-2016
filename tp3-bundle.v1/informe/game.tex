
\subsection{Game}

\label{sec:game}

En esta seccion se describira el estado del juego que se representa en pantalla.
La mayoria del estado del juego se describe mediante la estructura \verb|jugador|. Contiene la posicion en el mapa del jugador, los puntos obtenidos y las vidas restantes.



\begin{verbatim}

typedef struct jugador_t{
	unsigned int tareas_restantes;
	unsigned short x;
	unsigned short y;
	unsigned short id;
	unsigned short puntos;
} jugador;

jugador jugadores[2];


\end{verbatim}


\subsubsection{Interaccion con el usuario}


Las siguientes rutinas hacen de interfaz con los eventos de teclado que podemos recibir. El kernel recibe las teclas pulsadas y decide el curso de accion. 
En caso de la tecla pulsada ser un "shift" se llama a la funcion \verb|game_lanzar| con el indice de jugador correspondiente (0 para la izquierda o 1 para la derecha). Ella se ocupa de verificar que el jugador tenga "vidas" (representadas mediante \verb|tareas_restantes|) y que haya lugar en el scheduler (solo se pueden correr 5 tareas por jugador). De ser asi, se resta una vida al jugador y se le pide al scheduler que lance una tarea en la posicion actual del jugador. En la seccion \ref{sec:sched} se define lo que implica lanzar una tarea,

Si las teclas pulsadas fueron las de movimiento, el kernel se encarga de transformarlas a un indice de jugador y una direccion. Representamos las direcciones con el siguiente tipo enumerado:

\begin{verbatim}

typedef enum direccion_e { IZQ = 0xAAA, DER = 0x441, ARB = 0xA33, ABA = 0x883  } direccion;

\end{verbatim}

La funcion \verb|game_mover_cursor|, dado un jugador y una direccion, se encarga de de modificar acordemente la posicion en el mapa de la estructura jugador.

\subsubsection{Interaccion con tareas}

Las siguiente son las rutinas que provee nuestro sistema para que las tareas interactuen con el estado del juego.
La funcion \verb|game_soy| recibe un parametro que determina si una tarea es roja, azul o verde. Se setea acoremente el parametro \verb|owner| de la tarea actual.
\verb|game_donde| recibe un puntero, donde depositamos las coordenadas $(x,y)$ de la tarea en el mapa.
\verb|game_mapear| recibe las coordenadas correspondientes a la pagina que la tarea desea mapear. Primero se verifica que sean coordenadas validas del mapa y se obtiene la direccion fisica de la pagina mediante \verb|xytofisica|. Si recibimos coordenadas invalidas, desalojamos a la tarea. Si no le pedimos a \verb|mmu_mapear_pagina| que mapee para la tarea actual la pagina deseada.


