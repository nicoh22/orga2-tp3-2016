\subsection{Gestión de tareas}

Tras definir y configurar las entradas en la IDT, construimos las estructuras que permiten saltar entre tareas.
Para albergar el contexto previo al primer salto, con el objetivo de partir desde un contexto limpio, creamos un TSS vacío, agregamos su descriptor a la GDT en el indice 9.
Hecho esto, lo apuntamos en el registro TR. De esta manera le estamos indicando al procesador en que segmento guardar el contexto una vez que saltemos a la primer tarea. 
Este tss, que denominamos \verb|tss_inicial|, se carga con todos sus registros en 0 ya que no se saltará a el y no se modificará más allá de este primer salto.

La tarea que ejecutaremos incialmente será la Idle. Para poder cambiar a su contexto, definimos su TSS y agregamos el descriptor correspondiente a la GDT en el indice 10.
En este segmento definimos los campos de la siguiente manera:


\begin{center}
\begin{tabular}{ |c| c | }
\hline
\verb|esp0| & Base de stack de kernel\\
\hline
\verb|ss0|  & Segmento de datos de kernel\\
\hline
\verb|cr3|  & Directorio de páginas de kernel\\
\hline
\verb|eip|  & \verb|0x1000|\\
\hline
\verb|esp|  & Base del stack de kernel\\
\hline
\verb|ebp|  & Base del stack de kernel\\
\hline
Registros prop. gral. & \verb|0|\\
\hline
\verb|eflags| & \verb|0x202|\\
\hline
\verb|cs| & Segmento de código de kernel\\
\hline
Segmentos de datos & Segmento de datos de kernel\\
\hline
\end{tabular}
\end{center}



Tanto para estos TSS, como para todos los de tareas lanzadas para el juego, usamos descriptores con nivel de privilegio cero y attributo presente en 1.\\\textbf{}

\label{sec-tss-tareasInfo}
Para gestionar el estado de las tareas del juego utilizamos una matriz, \verb|tareasInfo|. Esta tiene dimensiones $3\times15$, para acomodar todas las tareas en juego. La primer coordenada se corresponde con los primeros tres valores del enum \verb|task_type|, mientras que la segunda indica el indice o id de tarea. 
Siempre que se accede guardamos el recaudo de mantenernos solo en los indices válidos para cada tipo de acuerdo a la función \verb|task_type_max| que devuelve la máxima cantidad de tareas de la clase indicada. 
El tipo de esta matriz es una estructura que llamamos \verb|task_info|.
En ésta guardamos los siguientes campos:\\


\begin{center}
	\begin{tabular}{r p{0.7\textwidth} }
		\verb|alive| : & Flag para denotar si una tarea esta en juego. \\
		\verb|owner| : & Dueño de la tarea. Utiliza el tipo enumerado \verb|task_type|\\
		\verb|x| e \verb|y| : & Coordenadas en el mapa donde esta ubicada la tarea\\
		\verb|mapped_x| y \verb|mapped_y| : & Coordenadas en el mapa de la pagina mapeada por la tarea\\
		\verb|gdtIndex| : & Indice de la gdt correspondiente a esta tarea\\
	\end{tabular}
\end{center}
\vspace{10pt}

Para manejar los TSSs de las tareas del juego optamos por una matriz, llamada \verb|tss_directory|, de la estructura \verb|tss|. Esta matriz tiene las mismas dimensiones que \verb|tareasInfo|, de manera que los indices de cada tss se corresponden con los de la tarea cuyo contexto representa. 
\\

Utilizamos también estos indices para definir en que posición de la GDT ubicamos el descriptor de TSS, correspondiendo cada posición válida en la matriz con una ubicación en la GDT. De esta manera, partiendo del indice 11 de la tabla de descriptores globales, los primeros 15 consecutivos corresponden a las tareas sanas, los siguientes 5 a tareas del virus A y los últimos 5 a tareas del virus B.
En función de esto, para cada tarea de \verb|tareasInfo| calculamos su posición en la GDT con la siguiente formula: $ 11 + offset(tipo) + indice$.\\


Basándonos en estas variables y estructuras, al lanzar una tarea creamos su contexto a partir del tipo de tarea, su indice en \verb|tareasInfo|, su indice en la GDT, y la dirección de memoria física correspondiente a su posición en el mapa. 
Para ello generamos un directorio de paginas siguiendo el procedimiento descrito en la sección \ref{sec-paginacion-tareas} y obtenemos una página de memoria libre para la pila de nivel 0 de la tarea. Luego cargamos su descriptor en la entrada de GDT indicada y llenamos la tss ubicada en \verb|tss_directory[tipo][indice]| con los siguientes datos:\\


\begin{center}
	\begin{tabular}{ |c| c | }
		\hline
		\verb|esp0| & Página de memoria libre\\
		\hline
		\verb|ss0|  & Segmento de datos de kernel\\
		\hline
		\verb|cr3|  & Directorio de páginas de la tarea\\
		\hline
		\verb|eip|  & \verb|0x1000|\\
		\hline
		\verb|esp|  & \verb|0x8001000|\\
		\hline
		\verb|ebp|  & \verb|0x8001000|\\
		\hline
		Registros prop. gral. & \verb|0|\\
		\hline
		\verb|eflags| & \verb|0x202|\\
		\hline
		\verb|cs| & Segmento de código de usuario\\
		\hline
		Segmentos de datos & Segmento de datos de usuario\\
		\hline
	\end{tabular}
\end{center}


\label{sec:tss-tareasInfo-init}
Al inicializar las estructuras del juego y scheduler, dejamos en 0 todos los campos de cada entrada en \verb|tareasInfo|. Hecho esto se lanzan las 15 tareas sanas en posiciones al azar dentro del mapa usando el procedimiento que describimos en esta sección. A continuación se detalla cómo implementamos los saltos entre tareas y utilizamos las estructuras que introducimos aquí.






\begin{comment}
4.6.
Ejercicio 6


a) Definir las entradas en la GDT que considere necesarias para ser usadas como descriptores
de TSS. Minimamente, una para ser utilizada por la tarea inicial y otra para la tarea
Idle. Sugerencia: Hacer una función para obtener entradas libres en la gdt.


b) Completar la entrada de la TSS de la tarea Idle con la información de la tarea Idle. Esta
información se encuentra en el archivo TSS.C. La tarea Idle se encuentra en la dirección
0x00010000. La pila se alojará en la misma dirección que la pila del kernel y será mapeada
con identity mapping. Esta tarea ocupa 1 pagina de 4KB y debe ser “mapeada” con identity
mapping. Además la misma debe compartir el mismo CR3 que el kernel.


c) Construir una función que complete una TSS libre con los datos correspondientes a una
tarea. El código de las tareas se encuentra a partir de la dirección 0x00011000 ocupando
una pagina de 4kb cada una según indica la figura 1. Para la dirección de la pila de
nivel 3 se debe utilizar el mismo espacio de la tarea, la misma crecerá desde la base de
la tarea. Para el mapa de memoria se debe construir uno nuevo utilizando la función
mmu inicializar dir tarea. Además, tener en cuenta que cada tarea utilizará una pila
distinta de nivel 0, para esto se debe pedir una nueva pagina libre a tal fin.


d) Completar la entrada de la GDT correspondiente a la tarea inicial.


e) Completar la entrada de la GDT correspondiente a la tarea Idle.


f) Escribir el código necesario para ejecutar la tarea Idle, es decir, saltar intercambiando las
TSS, entre la tarea inicial y la tarea Idle.

\end{comment}