\subsection{Paginación}
\label{sec-paginacion} 

Para el manejo de páginas libres utilizamos el puntero \verb|proxima_pagina_libre| que denota la dirección de la próxima página libre, al que inicializamos apuntando al inicio del área libre, en la dirección \textit{0x100000}.
Administramos este puntero con la función \verb|mmu_proxima_pagina_fisica_libre|, que devuelve un puntero a la proxima página e incrementa el valor de \verb|proxima_pagina_libre| en 4KB.


{\LARGE \textbf{TODO: Explicar funciones mapear y desmapear}}\\

\subsubsection{Directorio de kernel}
Siguiendo las indicaciones del enunciado, para el directorio del kernel mapeamos las direcciones \textit{0x000000} a \textit{0x3FFFFF} usando \textit{identity mapping}.
Para lograrlo en primer lugar inicializamos los primeros 4KB de memoria a partir de la dirección \textit{0x27000} con ceros, donde luego definimos el directorio de páginas del kernel.
Ubicamos la primer página del directorio en la dirección \textit{0x28000}, con atributos de lectura y escritura, nivel de privilegios cero y el bit presente activo.
Luego, para mapear el sección de memoria pedida, llenamos la tabla con las direcciones de los primeros 1024 bloques de 4KB de memoria física. 
De esta manera definimos la última entrada de la tabla con la dirección base 0x3FF000, permitiendo direccionar las siguientes 4096-1 direcciones. 
Al igual que la definición de la tabla de páginas en el directorio, cada página fue definida con atributos de lectura y escritura, privilegios de kernel y bit presente activo.

Teniendo armado un directorio de páginas con una tabla de páginas, habilitamos paginación moviendo la dirección del directorio a CR3 y levantando el bit correspondiente en CR0.\\


\subsubsection{Directorios de tareas}
\label{sec-paginacion-tareas} 
Controlamos la inicialización de directorios para tareas con la función \verb|mmu_inicializar_dir_tarea|. 
Para ella requerimos los parámetros \verb|tipo|, que indica el tipo de tarea se esta mapeando, y \verb|fisica|, que define la dirección de memoria física dentro del mapa que debemos mapear. 
Denotamos el tipo de cada tarea con el enum \verb|task_type|, cuyos valores para tareas sanas, de virus A y B son: \verb|H_type(0)|, \verb|A_type(1)| y \verb|B_type(2)|.

En esta función debemos copiar el código adecuado y construir el directorio y páginas necesarias para que una tarea encuentre sus instrucciones a partir de la dirección virtual \textit{0x8000000} y pueda mapear otra página en la posición en el mapa a la dirección virtual \textit{0x8001000}.
Para ello usamos el directorio apuntado por CR3 al momento en que se llamó la función para copiar el código, mapeando la dirección física con identity mapping temporalmente usando atributos de kernel y escribiendo el código de la tarea. Obtenemos éste último calculando el desplazamiento a partir de la dirección \textit{0x10000}, que resulta del producto entre el tamaño de página y el orden en memoria correspondiente al tipo de tarea.
Una vez escrito el código, desmapeamos la página usada del directorio actual y construimos el de la tarea. 
Con este objetivo obtenemos una página para el directorio, que inicializamos con ceros, y seguimos el mismo procedimiento descrito para el directorio de kernel.
Una vez mapeada el area de kernel en el directorio de la tarea, vinculamos la posición física indicada a la dirección virtual 0x8000000. 
Consideramos que en un principio la tarea no esta atacando (mapeando la dirección) a otra tarea, por lo que también vinculamos la posición virtual 0x8001000 a la misma dirección física.\\


