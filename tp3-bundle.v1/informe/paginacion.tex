\subsection{Paginación}

Siguiendo las indicaciones del enunciado, mapeamos las direcciones \textit{0x000000} a \textit{0x3FFFFF} usando \textit{identity mapping}.
Para lograrlo en primer lugar inicializamos los primeros 4KB de memoria a partir de la dirección \textit{0x27000} con ceros, donde luego definimos el directorio de páginas del kernel.
Ubicamos la primer página del directorio en la dirección \textit{0x28000}, con atributos de lectura y escritura, nivel de privilegios cero y el bit presente activo.
Luego, para mapear el sección de memoria pedida, llenamos la tabla con las direcciones de los primeros 1024 bloques de 4KB de memoria física. 
De esta manera definimos la última entrada de la tabla con la dirección base 0x3FF000, haciendo direccionables las siguientes 4096-1 direcciones. 
Al igual que la definición de la tabla de páginas en el directorio, cada página fue definida con atributos de lectura y escritura, privilegios de kernel y bit presente activo.

Teniendo armado un directorio de páginas con una tabla de páginas definida, habilitamos paginación moviendo la dirección del directorio a CR3 y levantando el bit correspondiente en CR0.\\

Para el manejo de páginas libres utilizamos el puntero \verb|proxima_pagina_libre| que denota la dirección de la próxima página libre, al que inicializamos apuntando al inicio del área libre, en la dirección \textit{0x100000}.
Administramos este puntero con la función \verb|mmu_proxima_pagina_fisica_libre|, que devuelve un puntero a la proxima página e incrementa el valor de \verb|proxima_pagina_libre| en 4KB.

Controlamos la inicialización de directorios para tareas con la función \verb|mmu_inicializar_dir_tarea|. 
Para ella requerimos los parámetros \verb|int tipo|, que indica cual de los tres tipos de tarea se esta mapeando (sana, A o B), y \verb|int fisica|, que define la dirección de memoria física dentro del mapa que debemos mapear. 


En esta función mapeamos la posicion con identity mapping al directorio apuntado por cr3 con permisos de kernel para copiar el código. 
Una vez hecho esto la desmapeamos del directorio actual y vinculamos la posicion física en el mapa a la dirección virtual 0x8000000 en las tablas de paginación de la tarea que inicializamos. 
Consideramos que en un principio la tarea no esta atacando (mapeando la dirección) a otra tarea, por lo que también vinculamos la posición virtual 0x8001000 a la misma dirección física.\\


{\LARGE \textbf{TODO: Explicar funciones mapear y desmapear}}

