% ******************************************************** %
%              TEMPLATE DE INFORME ORGA2 v0.1              %
% ******************************************************** %
% ******************************************************** %
%                                                          %
% ALGUNOS PAQUETES REQUERIDOS (EN UBUNTU):                 %
% ========================================
%                                                          %
% texlive-latex-base                                       %
% texlive-latex-recommended                                %
% texlive-fonts-recommended                                %
% texlive-latex-extra?                                     %
% texlive-lang-spanish (en ubuntu 13.10)                   %
% ******************************************************** %


\documentclass[a4paper]{article}
\usepackage[spanish]{babel}
\usepackage[utf8]{inputenc}
\usepackage{charter}   % tipografia
\usepackage{graphicx}
\usepackage{comment}
\usepackage{tabularx}
\usepackage{paralist} %itemize inline

%\usepackage{float}
%\usepackage{amsmath, amsthm, amssymb}
%\usepackage{amsfonts}
%\usepackage{sectsty}
%\usepackage{charter}
%\usepackage{wrapfig}
%\usepackage{listings}
%\lstset{language=C}

% \setcounter{secnumdepth}{2}
\usepackage{underscore}
\usepackage{caratula}
\usepackage{url}
\usepackage[colorlinks,citecolor=black,filecolor=black,linkcolor=black,    urlcolor=black]{hyperref}



% ********************************************************* %
% ~~~~~~~~              Code snippets             ~~~~~~~~~ %
% ********************************************************* %

\usepackage{color} % para snipets de codigo coloreados
\usepackage{fancybox}  % para el sbox de los snipets de codigo

\definecolor{litegrey}{gray}{0.94}

\newenvironment{codesnippet}{%
	\begin{Sbox}\begin{minipage}{\textwidth}\sffamily\small}%
	{\end{minipage}\end{Sbox}%
		\begin{center}%
		\vspace{-0.4cm}\colorbox{litegrey}{\TheSbox}\end{center}\vspace{0.3cm}}



% ********************************************************* %
% ~~~~~~~~         Formato de las páginas         ~~~~~~~~~ %
% ********************************************************* %

\usepackage{fancyhdr}
\pagestyle{fancy}

%\renewcommand{\chaptermark}[1]{\markboth{#1}{}}
\renewcommand{\sectionmark}[1]{\markright{\thesection\ - #1}}

\fancyhf{}

\fancyhead[LO]{Sección \rightmark} % \thesection\ 
\fancyfoot[LO]{\small{Esquivel, Federico Nicolás, Hernandez, Nicolás, Karbopel, Rodrigo}}
\fancyfoot[RO]{\thepage}
\renewcommand{\headrulewidth}{0.5pt}
\renewcommand{\footrulewidth}{0.5pt}
\setlength{\hoffset}{-0.8in}
\setlength{\textwidth}{16cm}
%\setlength{\hoffset}{-1.1cm}
%\setlength{\textwidth}{16cm}
\setlength{\headsep}{0.5cm}
\setlength{\textheight}{25cm}
\setlength{\voffset}{-0.7in}
\setlength{\headwidth}{\textwidth}
\setlength{\headheight}{13.1pt}

\renewcommand{\baselinestretch}{1.1}  % line spacing

% ******************************************************** %


\begin{document}


\thispagestyle{empty}
\materia{Organización del Computador II}
\submateria{Segundo Cuatrimestre de 2014}
\titulo{Trabajo Práctico III}
\subtitulo{subtitulo del trabajo}
\integrante{Esquivel, Federico Nicolás}{915/12}{alt.juss@gmail.com}
\integrante{Hernandez, Nicolás}{XXX/XX}{nicoh22@hotmail.com}
\integrante{Karbopel, Rodrigo}{XXX/XX}{rok\_35@live.com.ar}

\maketitle
\newpage

\thispagestyle{empty}
\vfill


\thispagestyle{empty}
\vspace{3cm}
\tableofcontents
\newpage


%\normalsize
\newpage

\section{Objetivos generales}

%En este trabajo buscamos activar todos los sistemas para que un procesador trabaje en modo multi-tarea. Partiendo del estado en que el bootloader nos entrega el control, completaremos las estructuras necesarias para pasar a modo protegido, incorporar paginación, manejar excepciones e interrupciones, lanzar tareas y distribuir los recursos del procesador entre ellas.


\section{Desarrollo}

\textbf{TODO}: agregar introducción a lo que hicimos?

\subsection{Segmentación y manejo de excepciones}

De acuerdo a lo indicado en el enunciado, definimos segmentos de código y datos en la GDT a partir del indice 4, un par con privilegios de kernel y otro con privilegios de usuario. 
Direccionamos los primeros $878$MB de memoria con estos segmentos.
Para ello, establecimos la base de cada segmento en la dirección 0x0 y, para poder representar el número en 20 bits, calculamos el limite de cada uno en bloques de 4KB. 
Adicionalmente, definimos un segmento de datos con privilegios de kernel para la memoria de video, basado en la dirección 0xB8000. Dado que las dimensiones de la pantalla son 80x50 caracteres y que para representar cada uno se necesitan dos bytes, el tamaño de este segmento es de 8000 Bytes. En función de esto definimos el limite del segmento en 7999, su ultimo byte direccionable.\\

Resolvimos el manejo de excepciones definiendo los primeros veinte indices (de 0 a 19) en la tabla de descriptores de interrupción. 
Para ello utilizamos un macro en el cual referimos cada entrada $N$ al segmento de código de kernel, con el offset correspondiente su rutina de atención \_isr$N$. 
En cuanto los atributos de cada descriptor, dejamos en 1 el bit de presencia, asignamos nivel de privilegios de kernel y usamos el tipo \textit{interrupt}.

Para atender excepciones escribimos rutinas que muestran sus mensajes de error por pantalla.
Definimos cada \_isr$N$ con un macro en el cual obtenemos y mostramos el $N$-ésimo mensaje de un arreglo donde los guardamos previamente.\\


\subsection{Paginación}

\textbf{TODO: Explicar ej. 3 y 4}



\section{Conclusiones y trabajo futuro}


\end{document}

