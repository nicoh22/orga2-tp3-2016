\subsection{Modo debug}
\label{sec:debug}
De acuerdo a lo solicitado implementamos un modo debug que nos permite visualizar el estado de los registros y la pila al momento que se produce una excepción en una tarea, justo antes de que ésta sea desalojada.

Para ello utilizamos la variable \verb|debugState|, del tipo enumerado \verb|debug_state_type|, que define los estados del mecanismo de debug.
Estos estados son \verb|debugEnabled|, que indica que el estado de la próxima excepción será visualizado por pantalla, interrumpiendo momentaneamente el flujo del juego; \verb|debugInterrupted|, el estado al que se pasa cuando ocurre una excepción que debemos mostrar; y \verb|debugDisabled|, que denota el estado normal, en el cual no se detiene la ejecución de tareas al ocurrir una excepción. Este último es el estado inicial del juego.

Cuando se presiona la tecla 'y', usamos la función \verb|debug_switch_state| para controlar el estado al que debe pasar \verb|debugState|. En ella, si el modo debug esta deshabilitado, lo habilitamos. Si por el contrario, el valor de \verb|debugState| es \verb|debugEnabled|, lo deshabilitamos.
De estar en estado interrumpido, lo pasamos a habilitado para continuar la ejecución hasta la próxima excepción. 
En esta situación también indicamos que debemos redibujar la pantalla mostrar correctamente la sección del mapa que ocupa la tabla de visualización de registros.

De esta manera, el usuario solo es capaz de cambiar el estado de \verb|debugState| a habilitado o deshabilitado, dependiendo de su valor previo. Únicamente los handlers de excepciones pueden cambiar el estado a interrumpido. Logramos esto llamando a la función \verb|debug_set_interrupted| al entrar en uno de ellos. 
Esta función pasa \verb|debugStat| a estado interrumpido y devuelve $1$ si previamente el estado era habilitado. De lo contrario devuelve $0$, indicando que no debemos interrumpir la ejecución.

Si el handler recibe un $1$ de \verb|debug_set_interrupted|, encola todos los registros de propósito general, los selectores de segmento, registros de control y las últimas cuatro posiciones de la pila de la tarea. Hecho esto, llama a la función \verb|screen_imprimir_log|, que muestra en pantalla una tabla con toda esta información.
Al retornar mostramos el nombre de la excepción en la esquina superior izquierda, desalojamos la tarea causante y saltamos a la tarea idle.

De acuerdo a lo que mencionamos en la sección \ref{sec:sched}, mientras \verb|debugState| tenga el valor \verb|debugInterrupted|, no se producen cambios de tarea en la interrupción de reloj.

