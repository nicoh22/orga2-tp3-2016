\subsection{Scheduler de tareas}

\begin{comment}
4.7.
Ejercicio 7

a) Construir una función para inicializar las estructuras de datos del scheduler.


b) Crear la función sched proximo indice() que devuelve el ındice en la GDT de la próxima
tarea a ser ejecutada. Construir la rutina de forma devuelva una tarea de cada jugador
por vez según se explica en la sección 3.2

c) Modificar la rutina de la interrupción 0x66, para que implemente los tres servicios según
se indica en la sección 3.1.1.


d) Modificar el código necesario para que se realice el intercambio de tareas por cada ciclo de
reloj. El intercambio se realizará según indique la función sched proximo indice().


e) Modificar las rutinas de excepciones del procesador para que desalojen y destruyan a la
tarea que estaba corriendo y corran la próxima.


f) Implementar el mecanismo de debugging explicado en la sección 3.4 que indicará en pan-
talla la razón del desalojo de una tarea.


Nota: Se recomienda construir funciones en C que ayuden a resolver problemas como
convertir direcciones de el mapa a direcciones fısicas o buscar la proxima tarea a ejecutar.

\end{comment}


{\LARGE \textbf{TODO: Explicar como manejamos las tareas que estan corriendo (sched\_proximo\_indice)}}


\subsubsection{Desalojo de tareas}
\label{sec-desalojo}

