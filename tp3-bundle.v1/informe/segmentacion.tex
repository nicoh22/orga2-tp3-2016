
\subsection{Segmentación y activación de modo protegido}
\label{sec:segmentacion}

De acuerdo a lo indicado en el enunciado, definimos segmentos de código y datos en la GDT a partir del indice 4. El primer par con privilegios de kernel (DPL 0), con indices 4 y 5 respectivamente, y un segundo par de segmentos con privlegios de usuario (DPL 3) en los indices 6 y 7 de la GDT. 
Direccionamos los primeros $878$MB de memoria con estos segmentos.
Para ello establecimos la base de cada uno en la dirección \textit{0x0} y, para poder representar el límite, usamos granularidad de 4KB. De esta manera, $878$MB se corresponden a $224768$ bloques de 4KB. Como en el limite indicamos el último bloque direccionable, restamos uno a esta cantidad. Luego, el límite calculado para estos segmentos fue \textit{0x36DFF}.

Adicionalmente, definimos un segmento de datos con privilegios de kernel para la memoria de video, basado en la dirección \textit{0xB8000}, en el indice 8 de la GDT. 
Dado que las dimensiones de la pantalla son $80\times50$ caracteres y que para representar cada uno se necesitan dos bytes, el tamaño de este segmento es de $8000$ Bytes. 
En función de esto definimos el limite del segmento en $7999$, su ultimo byte direccionable. \\

Una vez que construimos estos segmentos en la GDT, la apuntamos en el registro GDTR y activamos modo protegido. Para ello seteamos el bit PE en CR0 y usamos un jump far para cargar el selector de segmento de código.
Tras el salto, configuramos los selectores de segmentos de datos (ds, ss, es, fs y gs), apuntándolos al segmento de datos de kernel que definimos en el índice 5 de la tabla de descriptores.
Asimismo, asignamos el valor \textit{0x2700} a los registros esp y ebp para configurar la pila de kernel.\\


Hecho esto, inicializamos la pantalla.
En pos de esto usamos el segmento de datos que definimos para el área de la pantalla, cargando su selector en gs. Luego nos desplazamos por el rango permitido por el segmento, escribiendo \textit{0x7020} en cada par de bytes hasta alcanzar el límite. Este valor corresponde a un espacio con fondo gris.\\



