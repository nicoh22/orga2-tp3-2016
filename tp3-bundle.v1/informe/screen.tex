\vspace{10pt}
\subsection{Pantalla}

Tratamos de mantener el código que maneja el estado de la pantalla lo más modular posible. Es decir, evitar leer/modificar estructuras ajenas  y que las demás secciones de nuestro código no tengan que conocer exactamente como esta compuesta la pantalla.\\

Logramos esto proveyendo desde  \verb|screen.c| funciones que realizan de mediadoras entre la memoria de video y las estructuras del juego. Estas funciones no reciben como parametros ninguna estructura, sino identificadores basicos que den la informacion necesaria. Cabe destacar también que, para mantener consistencia entre los datos mostradas en pantalla y el estado del juego, la mayoría de estas funciones son llamadas en la rutina de atención del reloj (ver \ref{subsec:int-reloj}).\\


\verb|screen_pintar_tarea|: Recibe tipo de tarea y el $x$ e $y$ que esta ocupa en el mapa (notar que esto es diferente al $x$ e $y$ que ocupa en la pantalla).\\

\verb|screen_pintar_jugador|: Recibe el indice de jugador (jugadorA o jugadorB) y su posicion en el mapa.\\

\verb|screen_pintar_mapeo_tarea|: Pinta la pagina extra de una tarea.Recibe posicion en el mapa de la pagina y tipo de tarea.\\

\verb|screen_limpiar_posicion|: Limpia una posición del mapa (la pinta de gris claro).\\

\verb|screen_actualizar_puntos|: Recibe los puntos de los jugadores y los imprime en pantalla.\\

\verb|screen_actualizar_vidas|: Recibe \verb|tareas_restantes| de ambos jugadores y los imprime.\\

\verb|screen_actualizar_reloj_tarea|: Recibe el tipo de tarea, el indice (este mismo indice se usa en el scheduler), si esta viva o no la tarea y el dueño actual de la tarea (quien la infecto). Se posee una matriz \verb|clock_State| que indica el ultimo estado de reloj de la tarea, se modifica este estado por el siguiente (este orden definido en el arreglo \verb|clock|) y se lo imprime en pantalla.\\

\verb|screen_imprimir_log|: Imprime el log correspondiente al modo debug. Recibe un puntero a la pila. De alli se extraen en orden los parametros a imprimir.\\

\newpage

